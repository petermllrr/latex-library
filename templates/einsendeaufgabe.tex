\documentclass[11pt]{article}

\usepackage{amsmath} % \usepackage[fleqn]{amsmath} for left aligning all equations
\usepackage{amsthm}
\usepackage{amssymb}
\usepackage[ngerman]{babel}	
\usepackage{csquotes}
\usepackage{enumerate}
\usepackage{geometry}
\usepackage{graphicx}
\usepackage{hyperref}
\usepackage{listings}
\usepackage{newtxtext}
\usepackage{newtxmath}
\usepackage[headsepline]{scrlayer-scrpage}
\usepackage{xcolor}

\geometry{a4paper,
	top = 30mm,
	right = 25mm,
	bottom = 30mm,
	left = 25mm}
\setlength{\parindent}{0em}
\setlength{\parskip}{1em}
\ihead{Einsendeaufgaben Kurseinheit 1}
\ohead{Max Mustermann (123456789)}
\MakeOuterQuote{"}
\lstset{basicstyle = \small\ttfamily,
    backgroundcolor=\color[rgb]{0.95, 0.95, 0.97},
    keywordstyle = \color{magenta},
    commentstyle = \color[rgb]{0, 0.6, 0},
    identifierstyle = \color{black},
    stringstyle = \color[rgb]{0.58, 0, 0.82},
    numbers = left,
    numberstyle = \small\ttfamily\color[rgb]{0.7, 0.7, 0.7},
    showstringspaces = false,
    frame = single,
    framerule = 0pt}
\hypersetup{colorlinks = true,
    linkcolor = blue,
    filecolor = blue,      
    urlcolor = blue}

\begin{document}

\section*{Template für Einsendeaufgaben}

Lorem ipsum dolor sit amet, consectetur adipiscing elit. Nullam nec
quam finibus, pharetra felis quis, pretium tortor. Etiam quis libero quam. Ut
fringilla lectus non nulla dapibus, ac maximus orci iaculis. Proin lectus erat,
luctus non neque eu, malesuada tempor massa. Aliquam erat volutpat. Fusce vitae
consequat odio. Nulla eget ligula facilisis, mollis lacus non, eleifend ante.

Sed vulputate sollicitudin erat, sit amet ultrices tortor suscipit sit amet.
Nullam et enim et mauris malesuada rutrum vel quis dolor. Phasellus dapibus
sollicitudin libero a luctus. Pellentesque habitant morbi tristique senectus et
netus et malesuada fames ac turpis egestas. Suspendisse vel fringilla massa.
Pellentesque luctus turpis at tellus auctor, id rutrum lorem viverra.

\href{https://google.com}{Ein Textlink}

\begin{align*}
    f(x) &= a^2 + 2ab + b^2 \\
         &= (a + b)^2
\end{align*}

\begin{flalign*}
    f(x) &= a^2 + 2ab + b^2 & \\
         &= (a + b)^2 &
\end{flalign*}

\begin{flalign*}
   && a = b+c \\
   && = 1+1 \\
   && = 2  
\end{flalign*}

\bigskip

\begin{lstlisting}[language = java]
class Example {
    public static void main(String[] args) {
        int i = 123;
        /* A code example */
    }
}
\end{lstlisting}

\begin{enumerate}[(a)]
    \item
    Aufzählung
    \begin{enumerate}[(i)]
        \item 
        Unterpunkt
        \item 
        Unterpunkt
    \end{enumerate}
    \item
    Aufzählung
\end{enumerate}

\end{document}
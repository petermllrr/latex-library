% Requires packages:
% texlive texlive-xetex latexmk texlive-lang-german texlive-latex-extra python3-pygments 
%
% This example requires the font "Inter" and "Jet Brains Mono" from Google fonts:
% https://fonts.google.com/specimen/Inter?query=inter
% https://fonts.google.com/specimen/JetBrains+Mono?query=jet
%
% Run
% ./create-pdf.sh
% to create a pdf

\documentclass{article}

\usepackage{fontspec}
\usepackage{geometry}
\usepackage[ngerman]{babel}	
\usepackage{csquotes}
\usepackage{graphicx}
\usepackage{fancyhdr}
\usepackage{enumerate}
\usepackage{minted}

\geometry{a4paper,
  top = 25mm,
  left = 20mm,
  bottom = 20mm,
  right = 20mm}
\pagestyle{fancy}
\fancyhead[LO,LE]{XXXX Kursname - Einsendeaufgaben X}
\fancyhead[RO,RE]{}
\title{XXXX Kursname - Einsendeaufgaben X}
\author{Peter Müller - 3274969}
\setmainfont{Inter}
\setmonofont[Scale=0.9]{JetBrains Mono}

\begin{document}

\setlength{\baselineskip}{1.5em}
\setlength{\parindent}{0em}
\setlength{\parskip}{1em}

\maketitle

\newpage

\section*{Aufgabe 1}

Ein einfaches Template welches lokal installierte Schriften nutzt. Laut Definition in Abb. 2 sind je zwei Punkte durch eine Komposition mit einem Segment verbunden. Teilobjekte einer Komposition dürfen nur mit einem einzigen Ganzes-Objekt assoziiert sein. In unserem Beispiel bedeutet das, dass ein Punkt niemals mit zwei verschiedenen Segmenten assoziiert sein darf. \textit{Punkt 3} ist jedoch sowohl Teilobjekt von \textit{Segment 1} und \textit{Segment 3}. Um das Problem zu beheben, müssen wir einen weiteres Punktobjekt erschaffen, damit beide Segmente mit jeweils zwei disjunkten Punktobjekten assoziiert sind.

\linespread{1}
\begin{minted}{c}
int gettoken()
{
    int c;
    state = 0;
    start_state = 0;
    while (TRUE) {
        switch (state) {
        case 11: c = nextchar();
                    if isdigit(c)      state = 12;
                    else if issign(c)  state = 13;
                    else               state = next_diagram();
                    break;
        case 12: c = nextchar();
                    if isdigit(c);     state = 12;
                    else               state = 14;
                    break;
        case 13: c = nextchar();
                    if isdigit(c)      state = 12;
                    else               state = next_diagram();
                    break;
        case 14: c = nextchar();
                    if isdigit(c)      state = 15;
                    if (c == 'E')      state = 17;
                    else               state = 16;
                    break;
    }
};
\end{minted}

Ein einfaches Template welches lokal installierte Schriften nutzt. Laut Definition in Abb. 2 sind je zwei Punkte durch eine Komposition mit einem Segment verbunden. Teilobjekte einer Komposition dürfen nur mit einem einzigen Ganzes-Objekt assoziiert sein. In unserem Beispiel bedeutet das, dass ein Punkt niemals mit zwei verschiedenen Segmenten assoziiert sein darf. \textit{Punkt 3} ist jedoch sowohl Teilobjekt von \textit{Segment 1} und \textit{Segment 3}. Um das Problem zu beheben, müssen wir einen weiteres Punktobjekt erschaffen, damit beide Segmente mit jeweils zwei disjunkten Punktobjekten assoziiert sind.

\end{document}
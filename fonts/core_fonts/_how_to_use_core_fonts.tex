\documentclass{article}

\begin{document}

\section*{Using LaTex NFSS Core Fonts}

\LaTeX\ provides basic core fonts, called the NFSS collection. The easiest and
recommended way to use them is to simply load them by package name (even though
they are included in \TeX).

\medskip

\begin{tabular}{ | l | l l l l |}
    \hline
    package     & roman     & sans serif    & typewriter      & formulas \\
    \hline & & & &\\
    \hline \\
\end{tabular}

\bigskip\noindent See the documentation \\
\texttt{http://ctan.ebinger.cc/tex-archive/macros/latex/required/psnfss/psnfss2e.pdf}

\begin{verbatim}
\renewcommand{\rmdefault}{...} % Sets default (serif) font
\renewcommand{\sfdefault}{...} % Sets sans-serif font
\renewcommand{\ttdefault}{...} % Sets monospaced font
\end{verbatim}

\renewcommand{\rmdefault}{ptm}
\renewcommand{\sfdefault}{phv}
\renewcommand{\ttdefault}{pcr}

\noindent\textrm{Serif font: lorem ipsum dolor sit amet, consectetur adipiscing
elit. Nullam nec quam finibus, pharetra felis quis, pretium tortor. Etiam quis
libero quam. Suspendisse vel fringilla massa. Pellentesque luctus turpis at
tellus auctor.}

\bigskip\noindent\textsf{Sans serif font: Suspendisse vel fringilla massa.
Pellentesque luctus turpis at tellus auctor, id rutrum lorem viverra. Lorem
ipsum dolor sit amet, consectetur adipiscing elit. Nullam nec quam finibus,
pharetra felis quis, pretium tortor.}

\begin{verbatim}
class MonoSpacedFont {
    public static void main(String[] args) {
        /* A code example */
        String[][] output = new String[3][3];
        for (int col = 0; col < input.length; col++) {
            for (int row = 0; row < input[0].length; row++) {
                output[col][row] = input[col][row];
            }
        }
    }
}
\end{verbatim}

\end{document}